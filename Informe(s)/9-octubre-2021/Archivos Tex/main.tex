\documentclass{article}
\usepackage[utf8]{inputenc}
\usepackage[spanish]{babel}
\usepackage{listings}
\usepackage{graphicx}
\graphicspath{ {Images/} }
\usepackage{cite}

\begin{document}

\begin{titlepage}
    \begin{center}
        \vspace*{1cm}
            
        \Huge
        \textbf{Proyecto Final}
            
        \vspace{0.5cm}
        \LARGE
        Informe Escrito
            
        \vspace{1.5cm}
            
        \textbf{Julian Taborda Ramirez}
        
        \vspace{0.5cm}
        
        \textbf{Samuel Ruiz Vargas}
            
        \vfill
            
        \vspace{0.8cm}
            
        \Large
        Informática II\\
        Universidad de Antioquia\\
        Medellín\\
        Octubre de 2021
            
    \end{center}
\end{titlepage}

\tableofcontents
\vspace*{1.2cm}

\newpage

\section{Planificación y Diseño}
        Iniciamos identificando nuestras tareas, una vez hecho esto comenzamos a dividir nuestro trabajo de manera que, pudiéramos trabajar eficientemente, para ello cada uno escogió las tareas con la que se sintiera más cómodo. Posteriormente realizamos un cronograma basándonos en la importancia y el tiempo requerido para cada tarea. El plan fue seguir el cronograma lo más fiel posible sin embargo siempre estuvo sujeto a cambios. 
        
        \vspace{0.3cm}
        
        Para el diseño decidimos utilizar un estilo similar a Mario Bros, haciendo uso de una pantalla principal en la cual se podrán escoger opciones tales como 1 jugador, multijugador, cargar partidas, nueva partida, etc. Posteriormente se iniciara el juego con una corta y simple historia, la jugabilidad constara de 3 niveles de plataformas comunes en los cuales se enfrentara a enemigos con distintas clases de ataques, además de ir desbloqueando las habilidades básicas con cada nivel completado, finalmente se enfrentara al Boss final del juego, lo cual le dará la conclusión a la historia.

\section{Avances Regulares}
    \subsection{9/10/2021}
        Este día fue la apertura del telón para comenzar nuestro proyecto final, aquí nos encargamos de lo relacionado a la planificación del proyecto así como a iniciar este informe. Además de esto creamos una pagina en Notion para mantenernos al tanto de las actividades y siempre estar organizados (Link al Notion en el README del repositorio).
        
    \subsection{11/10/2021}
        Día de corto trabajo, nos dedicamos a los primeros pasos en la interfaz y pantalla principal (aun falta), en este proceso encontramos varios inconvenientes por lo que intentaremos después haciendo uso de clases. Aun necesitamos investigar como hacer botones push dentro de la escena
    
\section{Modelamiento de las Clases}
    \subsection{}
        
        
\section{Estructura del Código}
    \subsection{}
    
    
\section{Problemas Presentados}
    \subsection{}

\end{document}
