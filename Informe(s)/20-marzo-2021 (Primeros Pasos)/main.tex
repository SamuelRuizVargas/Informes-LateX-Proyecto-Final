\documentclass{article}
\usepackage[utf8]{inputenc}
\usepackage[spanish]{babel}
\usepackage{listings}
\usepackage{graphicx}
\graphicspath{ {images/} }
\usepackage{cite}

\begin{document}

\begin{titlepage}
    \begin{center}
        \vspace*{1cm}
            
        \Huge
        \textbf{Los primeros pasos}
            
        \vspace{0.5cm}
        \LARGE
        Informatica II
            
        \vspace{1.5cm}
            
        \textbf{Samuel Ruiz Vargas}
        \vspace{0.5cm}
        
        \textbf{Julian David Taborda Ramirez}
            
        \vfill
            
        \vspace{0.8cm}
            
        \Large
        Proyecto Final\\
        Despartamento de Ingeniería Electrónica y Telecomunicaciones\\
        Universidad de Antioquia\\
        Marzo de 2021
            
    \end{center}
\end{titlepage}

\section{Cyberbug}
    \begin{flushleft}
    
    -    Juego de plataformas estático con enemigos.  Para avanzar de nivel el jugador deberá matar a un jefe por nivel. 
    \vspace*{1cm}
    
    -    La ambientación se va a diseñar de manera futurística (Cyberpunk), con cambios espacios-temporales.
    \vspace*{1cm}
    
    -    Los enemigos tendrán diferentes tipos de ataques.
    \vspace*{1cm}
    
    -    Para dos jugadores (A considerar). 
    \vspace*{1cm}
    
    -    Niveles de dificultad (A considerar). 
    \vspace*{1cm}
    
    -    Tiempo para pasar determinado nivel. 
    \vspace*{1cm}
    
    -    Contar una historia con referencias a otros video juegos, enfocada a la comedia con el fin de mantener al jugador entretenido. 
    \vspace*{1cm}
    
    -    Jugabilidad frenética. 
    \vspace*{1cm}
    
    -    El jugador tendrá diversas habilidades. 
    \vspace*{1cm}
    
    -    Selección de partidas cargadas. 
    \vspace*{1cm}
    
    -    Puntajes y records entre personas que jueguen.
    \vspace*{1cm}
    
    -    Sistemas de vidas limitadas.
    \vspace*{1cm}
    
    \end{flushleft}

\end{document}
